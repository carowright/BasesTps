\documentclass[10pt,a4paper]{article}
\usepackage[utf8]{inputenc} % para poder usar tildes en archivos UTF-8
\usepackage[spanish]{babel} % para que comandos como \today den el resultado en castellano
\usepackage{a4wide} % márgenes un poco más anchos que lo usual
\usepackage[conEntregas]{caratula}
\usepackage{ulem}
\usepackage{amsmath} 
\usepackage{amssymb}
\usepackage{fancybox}
\usepackage[usenames,dvipsnames]{color}
\usepackage{hyperref}
\usepackage{listings}
\usepackage{clrscode3e}
\usepackage{xcolor}
\usepackage{amsmath}
\usepackage{arydshln}

\hypersetup{
    colorlinks,
    citecolor=black,
    filecolor=black,
    linkcolor=black,
    urlcolor=black
}

\lstdefinestyle{customc}{
  belowcaptionskip=1\baselineskip,
  breaklines=true,
  frame=L,
  xleftmargin=\parindent,
  language=C,
  showstringspaces=false,
  basicstyle=\footnotesize\ttfamily,
  keywordstyle=\bfseries\color{green!40!black},
  commentstyle=\itshape\color{purple!40!black},
  identifierstyle=\color{blue},
  stringstyle=\color{orange},
}

\lstset{escapechar=@,style=customc}

\begin{document}

\titulo{Trabajo Práctico 1}
\subtitulo{[Primera entrega]}

\fecha{\today}

\materia{Bases de Datos}
\integrante{Fernandez, Esteban}{691/12}{esteban.pmf@gmail.com}
\integrante{Marta, Cristian G.}{079/12}{cristiangmarta@gmail.com}
\integrante{Podavini Rey, Martín Gastón}{483/12}{marto.rey2006@gmail.com}
\integrante{Wright, Carolina Rocío}{876/12}{wright.carolina@gmail.com}

\maketitle

\tableofcontents
\newpage

\section{Introducción}

\section{MER y MR}

\subsection{Modelo de Entidad Relación}

\subsection{Modelo Relacional}
En esta sección presentamos el diseño lógico, obtenido a partir del Modelo de Entidad Relación. 
\newline
\newline
Testimonio(\underline{id}, texto, hora, fecha, \dashuline{numeroPlacaPolicia})\\
	PK = CK = \{id\}\\ 
	FK = \{numeroPlacaPolicia\}\\ 
\newline
Evidencia(\underline{id}, fechaIngreso, descripcion, fechaEncuentro, horaEncuentro, fechaSellado, horaSellado, \dashuline{idCaso})\\ 
	PK = CK = \{id\} \\
	FK = \{idCaso\}\\ 
\newline
Custodia(\underline{id}, fecha, hora, comentario, \dashuline{idEvidencia}, \dashuline{numeroPlacaPolicia})\\ 
	PK = CK = \{id\} \\
	FK = \{idEvidencia, numeroPlacaPolicia\}\\ 
\newline
Direccion(\underline{id}, numero, calle)\\ 
	PK = CK = \{id\}\\ 
\newline
CasoCriminal(\underline{id}, fechaIngreso, fecha, hora, lugar, descripcion, \dashuline{nombreCategoria})\\ 
	PK = CK = \{id\}\\ 
	FK = \{nombreCategoria\}\\ 
\newline
Congelado(\dashuline{idCaso}, fecha, comentario)\\ 
	PK = CK = FK = \{idCaso\}\\ 
\newline
\textit{Congelado.idCaso} tiene que estar en \textit{CasoCriminal.id}
\newline
\newline
Descartado(\dashuline{idCaso}, fecha, motivo)\\ 
	PK = CK = FK = \{(idCaso)\}\\ 
\newline
\textit{Descartado.idCaso} tiene que estar en \textit{CasoCriminal.id}
\newline
\newline
Resuelto(\dashuline{idCaso}, fecha, descripcion, \dashuline{numeroPlacaPolicia})\\ 
	PK = CK = \{idCaso\}\\ 
	FK = \{idCaso, numeroPlacaPolicia\}\\ 	
\newline
\textit{Resuelto.idCaso} tiene que estar en \textit{CasoCriminal.id}
\newline
\newline
Categoria(\underline{nombre})\\ 
	PK = CK = \{nombre\}\\ 
\newline
Rol(\underline{nombre})\\ 
	PK = CK = \{nombre\}\\ 
\newline
Persona(\underline{dni}, fechaNacimiento, nombre, apellido, \dashuline{idDireccion})\\ 
	PK = CK = \{dni\}\\ 
	FK = \{idDireccion\}\\ 
\newline
Telefono(\underline{\underline{dni}}, \underline{numero})\\ 
	PK = CK = \{numero\}\\
	FK = \{dni\}\\ 
\newline
Evento(\underline{idEvento}, descripcion, hora, fecha, \dashuline{nombreRol, idCaso, dni})\\ 
	PK = CK  = \{idEvento\}\\ 
	FK = \{nombreRol, idCaso, dni\}\\
\newline
OficialDePolicia(\underline{numeroDePlaca}, fechaIngreso, numeroEscritorio, \dashuline{nombreRango}, \dashuline{nombreDepartamento})\\
	PK = CK = \{numeroDePlaca\}\\ 
	FK = \{nombreRango, nombreDepartamento\}\\
\newline
Servicio(\underline{nombre})\\ 
	PK = CK = \{nombre\}\\ 
\newline
Departamento(\underline{nombre}, lineaTelefonica, \dashuline{nombreLocalidad}, \dashuline{nombreSupervisa})\\ 
	PK = CK = \{nombre\}\\ 
	FK = \{nombreLocalidad, nombreSupervisa\}\\
\newline
LineaTelefonica(\underline{numero}, \underline{\underline{nombreDepartamento}})\\ 
	PK = CK = \{numero, nombreDepartamento\}\\ 
	FK = \{nombreDepartamento\}\\
\newline
Localicad(\underline{nombre})\\ 
	PK = CK = \{nombre\}\\ 
\newline
Rango(\underline{nombre})\\ 
	PK = CK = \{Nombre\}\\ 
\newline
Declara(\underline{\underline{idTestimonio, idCaso, dni}})\\ 
	PK = CK = FK = \{idTestimonio, idCaso, dni\}\\
\newline
Tiene(\underline{\underline{nombreRol, idCaso, dni}})\\ 
	PK = CK = FK = \{nombreRol, idCaso, dni\}\\ 
\newline
Culpable(\underline{\underline{dni, idCaso}})\\ 
	PK = CK = FK = \{dni, idCaso\}\\ 

\section{Diseño Físico}

\subsection{Inserción de datos}

\section{Código}

\section{Conclusion}

\end{document}