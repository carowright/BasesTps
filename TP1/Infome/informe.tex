\documentclass[10pt,a4paper]{article}
\usepackage[utf8]{inputenc} % para poder usar tildes en archivos UTF-8
\usepackage[spanish]{babel} % para que comandos como \today den el resultado en castellano
\usepackage{a4wide} % márgenes un poco más anchos que lo usual
\usepackage[conEntregas]{caratula}
\usepackage{ulem}
\usepackage{amsmath} 
\usepackage{amssymb}
\usepackage{fancybox}
\usepackage[usenames,dvipsnames]{color}
\usepackage{hyperref}
\usepackage{listings}
\usepackage{clrscode3e}
\usepackage{xcolor}
\usepackage{amsmath}
\usepackage{arydshln}

\hypersetup{
    colorlinks,
    citecolor=black,
    filecolor=black,
    linkcolor=black,
    urlcolor=black
}

\lstdefinestyle{customc}{
  belowcaptionskip=1\baselineskip,
  breaklines=true,
  frame=L,
  xleftmargin=\parindent,
  language=C,
  showstringspaces=false,
  basicstyle=\footnotesize\ttfamily,
  keywordstyle=\bfseries\color{green!40!black},
  commentstyle=\itshape\color{purple!40!black},
  identifierstyle=\color{blue},
  stringstyle=\color{orange},
}

\lstset{escapechar=@,style=customc}

\begin{document}

\titulo{Trabajo Práctico 1}
\subtitulo{[Primera entrega]}

\fecha{\today}

\materia{Bases de Datos}
\integrante{Fernandez, Esteban}{691/12}{esteban.pmf@gmail.com}
\integrante{Marta, Cristian G.}{079/12}{cristiangmarta@gmail.com}
\integrante{Podavini Rey, Martín Gastón}{483/12}{marto.rey2006@gmail.com}
\integrante{Wright, Carolina Rocío}{876/12}{wright.carolina@gmail.com}

\maketitle

\tableofcontents
\newpage

\section{Introducción}
El problema a resolver mediante el uso de bases de datos relacional es el del registro de los Casos Criminales en un sitio. Para llevar cuenta de ellos se tiene en cuenta los diversos aspectos que los componen, por ejemplo, los oficiales de policia (quienes pertenecen a sus correspondientes departamentos de policia los cuales fueron modelados tambien) encargados de documentar y resolver los crimenes junto a las demas personas que participaron de alguna manera (con sus respectivos roles) en el caso que se registra. Estos tendrán información asociada que los identifica y describe. \\
Los casos, como entidad, pueden encontrarse en diferentes estados (congelado, descartados, resuelto etc) y en cada uno de ellos se proveerá diferente tipo información ligada al estado en si del caso. \\
La utilización del modelo relacional nos permitirá  administrar esta información de la manera mas modular y ordenada posible, permitiéndonos así poder asociar, consultar y mantener la información de las diferentes entidades que conforman el registro de los Casos Criminales de una manera eficiente aprovechando el motor de las bases de datos relacionales SQL.
\section{MER y MR}

\subsection{Modelo de Entidad Relación}

\subsection{Modelo Relacional}
En esta sección presentamos el diseño lógico, obtenido a partir del Modelo de Entidad Relación. 
\newline
\newline
Testimonio(\underline{id}, texto, FechaHora, \dashuline{idCaso}, \dashuline{dni}, \dashuline{numeroPlacaPolicia})\\
	PK = CK = \{id\}\\ 
	FK = \{numeroPlacaPolicia, idCaso, dni\}\\ 
\newline
Evidencia(\underline{id}, fechaIngreso, descripcion, fechaHoraEncuentro, fechaHoraSellado, \dashuline{idCaso})\\ 
	PK = CK = \{id\} \\
	FK = \{idCaso\}\\ 
\newline
Custodia(\underline{id}, fechaHora, comentario, \dashuline{idEvidencia}, \dashuline{numeroPlacaPolicia})\\ 
	PK = CK = \{id\} \\
	FK = \{idEvidencia, numeroPlacaPolicia\}\\ 
\newline
Direccion(\underline{id}, numero, calle)\\ 
	PK = CK = \{id\}\\ 
\newline
CasoCriminal(\underline{id}, fechaIngreso, fechaHora, lugar, descripcion, estado, \dashuline{nombreCategoria})\\ 
	PK = CK = \{id\}\\ 
	FK = \{nombreCategoria\}\\ 
\newline
Congelado(\dashuline{idCaso}, fecha, comentario)\\ 
	PK = CK = FK = \{idCaso\}\\ 
\newline
\textit{Congelado.idCaso} tiene que estar en \textit{CasoCriminal.id}
\newline
\newline
Descartado(\dashuline{idCaso}, fecha, motivo)\\ 
	PK = CK = FK = \{(idCaso)\}\\ 
\newline
\textit{Descartado.idCaso} tiene que estar en \textit{CasoCriminal.id}
\newline
\newline
Resuelto(\dashuline{idCaso}, fecha, descripcion, \dashuline{numeroPlacaPolicia})\\ 
	PK = CK = \{idCaso\}\\ 
	FK = \{idCaso, numeroPlacaPolicia\}\\ 	
\newline
\textit{Resuelto.idCaso} tiene que estar en \textit{CasoCriminal.id}
\newline
\newline
Pendiente(\dashuline{idCaso})\\ 
  PK = CK = FK = \{idCaso\}\\ 
\newline
\textit{Pendiente.idCaso} tiene que estar en \textit{CasoCriminal.id}
\newline
\newline
Investiga(\dashuline{idCaso}, \dashuline{numeroPlacaPolicia}, esInvPrincipal?)\\ 
  PK = CK = FK = \{idCaso, numeroPlacaPolicia\}\\ 
\newline
\newline
Categoria(\underline{nombre})\\ 
	PK = CK = \{nombre\}\\ 
\newline
Rol(\underline{nombre})\\ 
	PK = CK = \{nombre\}\\ 
\newline
Persona(\underline{dni}, fechaNacimiento, nombre, apellido, \dashuline{idDireccion})\\ 
	PK = CK = \{dni\}\\ 
	FK = \{idDireccion\}\\ 
\newline
Telefono(\underline{\underline{dni}}, \underline{numero})\\ 
	PK = CK = \{numero\}\\
	FK = \{dni\}\\ 
\newline
Evento(\underline{idEvento}, descripcion, fechaHora, \dashuline{nombreRol, idCaso, dni})\\ 
	PK = CK  = \{idEvento\}\\ 
	FK = \{nombreRol, idCaso, dni\}\\
\newline
OficialDePolicia(\underline{numeroDePlaca}, fechaIngreso, numeroEscritorio, \dashuline{nombreRango}, \dashuline{nombreDepartamento})\\
	PK = CK = \{numeroDePlaca\}\\ 
	FK = \{nombreRango, nombreDepartamento\}\\
\newline
Servicio(\underline{nombre})\\ 
	PK = CK = \{nombre\}\\ 
\newline
Departamento(\underline{nombre}, lineaTelefonica, \dashuline{nombreLocalidad}, \dashuline{nombreSupervisa})\\ 
	PK = CK = \{nombre\}\\ 
	FK = \{nombreLocalidad, nombreSupervisa\}\\
\newline
LineaTelefonica(\underline{numero}, \underline{\underline{nombreDepartamento}})\\ 
	PK = CK = \{numero, nombreDepartamento\}\\ 
	FK = \{nombreDepartamento\}\\
\newline
Localicad(\underline{nombre})\\ 
	PK = CK = \{nombre\}\\ 
\newline
Rango(\underline{nombre})\\ 
	PK = CK = \{Nombre\}\\ 
\newline
Participa(\underline{\underline{nombreRol, idCaso, dni}})\\ 
	PK = CK = FK = \{nombreRol, idCaso, dni\}\\ 
\newline
Culpable(\underline{\underline{dni, idCaso}})\\ 
	PK = CK = FK = \{dni, idCaso\}\\ 

\section{Diseño Físico}
Creación de las tablas.

\begin{verbatim}
CREATE TABLE Direcciones(
  id integer primary key,
  numero integer not null,
  calle varchar(30) not null 
);

CREATE TABLE Personas(
  dni integer primary key,
  fecha_nacimiento date not null,
  nombre varchar(30) not null,
  apellido varchar(30) not null,
  direccion_id integer not null,
  foreign key (direccion_id) references Direcciones(id)
);

CREATE TABLE Telefonos(
  numero integer primary key,
  persona_dni integer not null,
  tipo varchar(30) not null,
  foreign key (persona_dni) references Personas(dni)
);

CREATE TABLE Localidades(
  nombre varchar(30) primary key
);

CREATE TABLE Departamentos(
  nombre varchar(30) primary key,
  nombre_localidad varchar(30) not null,
  supervisado_por_departamento varchar(30),
  foreign key (nombre_localidad) references Localidades(nombre),
  foreign key (supervisado_por_departamento) references Departamentos(nombre)
);

CREATE TABLE Lineas_Telefonicas(
  numero int,
  nombre_departamento varchar(30),
  PRIMARY KEY (numero, nombre_departamento),
  foreign key (nombre_departamento) references Departamentos(nombre)
);

CREATE TABLE Rangos(
  nombre varchar(30) primary key
);

CREATE TABLE Servicios(
  nombre varchar(30) primary key
);

CREATE TABLE Oficiales_De_Policia(
  persona_dni integer not null,
  numero_de_placa integer primary key,
  fecha_de_ingreso date not null,
  numero_de_escritorio integer not null,
  nombre_rango varchar(30) not null,
  nombre_departamento varchar(30) not null,
  foreign key (persona_dni) references Personas(dni),
  foreign key (nombre_rango) references Rangos(nombre),
  foreign key (nombre_departamento) references Departamentos(nombre)
);

CREATE TABLE Categorias(
  nombre varchar(30) primary key
);

CREATE TABLE Roles(
  nombre varchar(30) primary key
);

CREATE TABLE Casos_Criminales(
  id integer primary key,
  fecha_ingreso date not null,
  fecha datetime not null,
  lugar varchar(30) not null,
  descripcion varchar(255) not null,
  nombre_categoria varchar(30) not null,
  estado varchar(30) not null,
  foreign key (nombre_categoria) references Categorias(nombre)
);

CREATE TABLE Participa(
  caso_id integer,
  persona_dni integer,
  nombre_rol varchar(30) not null,

  PRIMARY KEY (caso_id, persona_dni),
  foreign key (caso_id)     references Casos_Criminales(id),
  foreign key (persona_dni) references Personas(dni),
  foreign key (nombre_rol)  references Roles(nombre)
);

CREATE TABLE Eventos(
  id integer primary key,
  caso_id integer,
  persona_dni integer,
  descripcion varchar(255) not null,
  fecha datetime not null,

  FOREIGN KEY (caso_id, persona_dni) REFERENCES Participa (caso_id, persona_dni)
);

CREATE TABLE Evidencias(
  id integer PRIMARY KEY,
  fecha_ingreso date not null,
  descripcion char(128) not null,
  fecha_encuentro datetime not null,
  fecha_sellado datetime not null,
  caso_id integer not null,
  FOREIGN KEY (caso_id) REFERENCES Casos_Criminales(id)
);

CREATE TABLE Testimonios(
  id integer PRIMARY KEY,
  caso_id integer,
  persona_dni integer,
  texto varchar(255) not null,
  fecha datetime not null,
  nro_placa_policia_a_cargo integer not null,
  FOREIGN KEY (caso_id, persona_dni) REFERENCES Participa (caso_id, persona_dni),
  FOREIGN KEY (nro_placa_policia_a_cargo) REFERENCES Oficiales_De_Policia(numero_de_placa)
);

CREATE TABLE Custodias(
  id integer PRIMARY KEY,
  fecha datetime not null,
  comentario varchar(255) not null,
  evidencia_id integer not null,
  nro_placa_policia_a_cargo integer not null,
  FOREIGN KEY (evidencia_id) REFERENCES Evidencias(id),
  FOREIGN KEY (nro_placa_policia_a_cargo) REFERENCES Oficiales_De_Policia(numero_de_placa)
);

CREATE TABLE Congelados(
  caso_id integer primary key,
  fecha date not null,
  comentario varchar(255) not null,
  FOREIGN KEY (caso_id) references Casos_Criminales(id)
);

CREATE TABLE Descartados(
  caso_id integer primary key,
  fecha date not null,
  motivo varchar(255) not null,
  FOREIGN KEY (caso_id) references Casos_Criminales(id)
);

CREATE TABLE Resueltos(
  caso_id integer primary key,
  fecha date not null,
  descripcion varchar(255) not null,
  nro_placa_policia_cerro integer not null,
  FOREIGN KEY (caso_id) references Casos_Criminales(id),
  FOREIGN KEY (nro_placa_policia_cerro) REFERENCES Oficiales_De_Policia(numero_de_placa)
);

CREATE TABLE Pendientes(
  caso_id integer primary key,
  FOREIGN KEY (caso_id) references Casos_Criminales(id)
);

CREATE TABLE Culpable(
  caso_id integer,
  persona_dni integer,

  PRIMARY KEY (caso_id, persona_dni),
  foreign key (caso_id)     references Resueltos(caso_id),
  foreign key (persona_dni) references Personas(dni)
);

CREATE TABLE Investiga(
  caso_id integer not null,
  nro_placa_policia integer not null,
  es_investigador_principal boolean not null,
  FOREIGN KEY (caso_id) REFERENCES Casos_Criminales(id),
  FOREIGN KEY (nro_placa_policia) REFERENCES Oficiales_De_Policia(numero_de_placa)
);

\end{verbatim}
\subsection{Inserción de datos}

\section{Código}
Implementamos las siguientes funcionalidades.

\begin{itemize}
\item Datos de las personas que fueron sospechosas.
\item Direcciones donde convivieron personas sospechosas de diferentes casos
\item Oficiales que participaron en la cadena de custodia de evidencias para más de un
caso.
\item La sucesión de eventos de personas involucradas en un caso.
\item Un ranking de oficiales exitosos, es decir los que cerraron mayor cantidad de casos
(resueltos).
\item Las ubicaciones de todas las evidencias de un caso.
\item La lista de oficiales involucrados en un caso.
\item Las categorías de casos ordenadas por cantidad de casos
\item Todos los testimonios de un caso dado
\item Para una categoría en particular listar, para cada uno de los casos, los testimonios
asociados
\end{itemize}
\section{Conclusion}

\end{document}